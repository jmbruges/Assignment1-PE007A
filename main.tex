\documentclass[a4paper, 11pt]{article}
\usepackage{comment} % enables the use of multi-line comments (\ifx \fi) 
\usepackage{lipsum} %This package just generates Lorem Ipsum filler text. 
\usepackage{fullpage} % changes the margin

\begin{document}
%Header-Make sure you update this information!!!!
\noindent
\large\textbf{Assignment \#1} \hfill \textbf{Javier Brug\'{e}s} \\
%\normalsize ECE 100-003 \hfill Teammates: Student1, Student2 \\
\today\\

\section*{Problem Statement}
Uppgift 1: Read Becher \& Trowler (2001) and reflect about the next points:

\begin{itemize}
	\item Create a map that describes your territory and your tribal affiliations within academy.
	\begin{itemize}
		\item What is your tribe or tribes?
		\item formal or informal?
		\item Which place do you occupy inside the tribe?
		\item Which cultural expressions can you identify?
		\item Which metaphors match your tribe?
	\end{itemize}
	\item Make a text about your findings in point 1 (max 1000 words).
	\item Blog about it latest 16 sept 23.00.
\end{itemize}


\section*{Notes}
- " The land exist without obsrver but landscape cannot: the scape is the projection of the human consciousness, the way the land is perceived and response to (Bowe.etal 1994) pg.16

-... Knowledge communities...pg.21

- Chapter 2: Point of departure
	- relationship between tribes and ideas(territories which range)
	
	\subsection*{The classification}
	Pure and applied, hard and soft - all related to the classification of knowledge


\section*{The knowledge}
In the book there is a table that cathegorize the different groups of knowledge into differenciate groups. 

But the important for me is how this groups are grouped. The selection is made by using a set of diferentiatiors, or characteristics by identifying a set of general questions
%  \ifx
	\begin{itemize}
		\item characteristics in the objects of enquiry
		\item the nature of knowledge wrought
		\item the relation between the researcher and knowledge
		\item enquiry procedure
		\item extend of truth trends and criteria for making them
		\item the results of research
\end{itemize}	
%  \fi
So in these descriptions, we can answer first what is my tribe? I am electrical engineer with specialization in optics and photonics, making me part of a multidiciplinary context. While pure electronic engineering discipline is hard to encompass outside the hard/applied, the issue concerning my field of optical engineering (after the specialization this is the more appropiate title that simplifies my field), this later has to deal not only with the engineering discipline but its roots (which I as researcher need to revise constantly) in physics. But here there is a disambiguation we need to do as well if I want first to revise my field. Optics is a branch of physics, but the just those concerning with theoretical physics are aware about optics a theoretical matter. With the development of optical theories, the last half of century optics itself has been prove to be a fundamental field and has derive its own subdisiplines, like photonics, optical engineering laser physics, and so on. Focusing back to optical engineering and asking the questions to cathegorize my field (or subfield) 1. it is not pureposive; 2. the knoledge is develope from a harder discipline like physics, but it is not concern on challenge the tehory but rather applied the acquire knowledge; 3. mastering the hard knowledge through advances in techniques and transform or create technologies; 4.


\subsection*{Question 2: Formal or informal?} 

Formal communication channel. Formal in all field of knowledge discipline. Traditional mentoring school structure. Formal way of collaboration (local and international).



\begin{thebibliography}{9}
\bibitem{Robotics} Fred G. Martin \emph{Robotics Explorations: A Hands-On Introduction to Engineering}. New Jersey: Prentice Hall.
\bibitem{Flueck}  Flueck, Alexander J. 2005. \emph{ECE 100}[online]. Chicago: Illinois Institute of Technology, Electrical and Computer Engineering Department, 2005 [cited 30
August 2005]. Available from World Wide Web: (http://www.ece.iit.edu/~flueck/ece100).
\end{thebibliography}

\end{document}
